\documentclass[twoside,10pt]{article}
\usepackage{amsmath,amsfonts,amsthm,fullpage}
%\usepackage{../mymath}
\usepackage{algorithm}
\usepackage{algorithmic}
\usepackage{graphicx}
\usepackage{color}



\def\argmin{\operatornamewithlimits{arg\, min}}

\begin{document}

\title{CS 7641 CSE/ISYE 6740 Homework 2}
\author{Parminder Bhatia}
\date{Deadline: 10/28 Tue, 1:30pm (before starting the class)
} \maketitle

\begin{itemize}
  \item Submit your answers as an electronic copy on T-square.
  \item No unapproved extension of deadline is allowed. Late
  submission will lead to 0 credit.
  \item Typing with Latex is highly recommended. Typing with MS Word is also okay.
  If you handwrite, try to be clear as much as possible. No credit may be given to unreadable handwriting.
  \item Explicitly mention your collaborators if any.
  \item Recommended reading: PRML\footnote{Christopher M. Bishop, Pattern Recognition and Machine
Learning, 2006, Springer.} Section 1.5, 1.6, 2.5, 9.2, 9.3

\end{itemize}


\section{EM for Mixture of Gaussians}
Mixture of $K$ Gaussians is represented as
\begin{equation}
p(x) = \sum_{k=1}^K \pi_k \mathcal{N}(x|\mu_k,
\Sigma_k),\label{eq:mixgauss}
\end{equation}
where $\pi_k$ represents the probability that a data point belongs
to the $k$th component. As it is probability, it satisfies $0 \le
\pi_k \le 1$ and $\sum_k \pi_k = 1$. In this problem, we are going
to represent this in a slightly different manner with explicit
latent variables. Specifically, we introduce 1-of-$K$ coding
representation for latent variables $z^{(k)} \in \mathbb{R}^K$ for
$k = 1, ..., K$. Each $z^{(k)}$ is a binary vector of size $K$, with
1 only in $k$th element and 0 in all others. That is,
\begin{align}
z^{(1)} &= [1; 0; ... ; 0]\nonumber\\
z^{(2)} &= [0; 1; ... ; 0]\nonumber\\
&\vdots\nonumber\\ z^{(K)} &= [0; 0; ... ; 1]\nonumber.
\end{align}
For example, if the second component generated data point $x^n$, its
latent variable $z^n$ is given by $[0; 1; ... ; 0] = z^{(2)}$. With
this representation, we can express $p(z)$ as
\begin{equation}
p(z) = \prod_{k=1}^K {\pi_k}^{z_k},\nonumber
\end{equation}
where $z_k$ indicates $k$th element of vector $z$. Also, $p(x|z)$
can be represented similarly as
\begin{equation}
p(x|z) = \prod_{k=1}^K \mathcal{N}(x|\mu_k,
\Sigma_k)^{z_k}.\nonumber
\end{equation}
By the sum rule of probability, \eqref{eq:mixgauss} can be
represented by
\begin{equation}
p(x) = \sum_{z \in Z} p(z) p(x|z).\label{eq:mixgauss2}
\end{equation}
where $Z = \{z^{(1)}, z^{(2)}, ..., z^{(K)}\}$.

\subsubsection*{(a) Show that \eqref{eq:mixgauss2} is equivalent to
\eqref{eq:mixgauss}. [5 pts]}
Now, by substituting $p(z)$ and $p(x|z)$ in $p(x)$, we get:

\begin{equation} \nonumber
\begin{split}
p(x) &= \sum_{z \in Z} p(z) p(x|z) \\
&= \sum_{z \in Z} \prod_{k=1}^K {\pi_k}^{z_k} \prod_{k=1}^K \mathcal{N}(x|\mu_k,\Sigma_k)^{z_k} \\
&= \sum_{z \in Z} \prod_{k=1}^K \left( \pi_k \mathcal{N}(x|\mu_k,\Sigma_k)\right)^{z_k}
\end{split}
\end{equation}

 For each $z \in Z$, only one of its entry is equal to 1, and rest of elements in the vector are equal to 0. 
 Thus, for all the vectors , there exist a particular $k \in 1 \dots K$ such that $z_k = 1$ and is 0 for the other values of $k$.  So for a particular $p(z)$ or$p(x/z)$, only contribution would come from  that element in the vector which is 1, for all the others, number raise to power zero would fetch 1.
\\

Therefore, from this explanation, we get:

\begin{equation} \nonumber
\begin{split}
p(x) &= \sum_{z \in Z} \left( \pi_1 \mathcal{N}(x|\mu_1,\Sigma_1)\right)^{z_1} \dots \left( \pi_K \mathcal{N}(x|\mu_K,\Sigma_K)\right)^{z_K} \\
&= \sum_{k=1}^{K} \pi_k \mathcal{N}(x|\mu_k,\Sigma_k)
\end{split}
\end{equation}

Thus, \eqref{eq:mixgauss2} is equivalent to \eqref{eq:mixgauss} \\
\subsubsection*{(b) In reality, we do not know which component each
data point is from. Thus, we estimate the responsibility
(expectation of $z_k^n$) in the E-step of EM. Since $z_k^n$ is either $1$ or $0$, its expectation is the
probability for the point $x_n$ to belong to the component $z_k$. In
other words, we estimate $p(z_k^n|x_n)$. Derive the formula for this
estimation by using Bayes rule. Note that, in the E-step, we assume all other parameters, i.e. $\pi_k$, $\mu_k$,
and $\Sigma_k$, are fixed, and we want to express $p(z_k^n|x_n)$ as a function of these fixed parameters. [10 pts]}

Applying the Bayes Rule, we get:

\begin{equation} \nonumber
\begin{split}
p(z_k^n|x_n) &= \frac{p(x_n|z_k^n) p(z_k^n)}{p(x_n)}
\end{split}
\end{equation}

Note that $z_k^n$ is a point and not a vector , so can't be used with above formulas directly.
In the 1-of-$K$ coding representation, $z_k^n$ will  thus be represented by a vector where the $k^{th}$ element in the vector will be equal to 1 and the rest will be zero, i.e., [0,0,\dots,1,\dots,0,0]  which is same  as $z^{(k)}$\\

Therefore, $p(x_n|z_k^n)$,  $p(z_k^n)$ and $p(x_n)$ are given by:

\begin{equation} \nonumber
\begin{split}
p(x_n|z_k^n) &= \prod_{k'=1}^K \mathcal{N}(x_n|\mu_k, \Sigma_k)^{{(z_k^n)}_{k'}} \\
p(z_k^n) &= \prod_{k'=1}^K {\pi_k}^{{(z_k^n)}_{k'}}
\end{split}
\end{equation}
Note that ${{(z_k^n)}_{k'}} $ is basically same as $z^{(k)}$ 
After substituting these equations into the equation for $p(z_k^n|x_n)$ 
\begin{equation} \nonumber
\begin{split}
p(z_k^n|x_n) &= \frac{p(x_n|z_k^n) p(z_k^n)}{p(x_n)} \\
&= \frac{\prod_{k'=1}^K \mathcal{N}(x_n|\mu_k, \Sigma_k)^{{(z_k^n)}_{k'}} \prod_{k'=1}^K {\pi_k}^{{(z_k^n)}_{k'}}}{ \sum_{k'=1}^K \pi_{k'} \mathcal{N}(x_n|\mu_{k'},\Sigma_{k'})} \\
&= \frac{\prod_{k'=1}^K \left({\pi_k} \mathcal{N}(x_n|\mu_k, \Sigma_k)\right)^{{(z_k^n)}_{k'}}}{ \sum_{k'=1}^K \pi_{k'} \mathcal{N}(x_n|\mu_{k'},\Sigma_{k'})}
\end{split}
\end{equation}

Note that ${{(z_k^n)}_{k'}} $ is basically same as $z^{(k)}$ 
Thus, in 1-of-$K$ coding representation, only one of its entry is equal to 1, and rest of them are equal to 0. Thus, there exist a particular $k' \in 1 \dots K$ such that ${(z_k^n)}_{k'} = 1$ and is 0 for the other values of $k$. \\

Therefore,  we get:

\begin{equation} \nonumber
\begin{split}
p(z_k^n|x_n) &= \frac{\prod_{k'=1}^K \left({\pi_k} \mathcal{N}(x_n|\mu_k, \Sigma_k)\right)^{z^{(k')}}}{ \sum_{j=1}^K \pi_{j} \mathcal{N}(x_n|\mu_{j},\Sigma_{j})} \\
&= \frac{ \pi_k \mathcal{N}(x_n|\mu_k, \Sigma_k)}{ \sum_{j=1}^K \pi_{j} \mathcal{N}(x_n|\mu_{j},\Sigma_{j})}
\end{split}
\end{equation}
\subsubsection*{(c) In the M-Step, we re-estimate parameters $\pi_k$, $\mu_k$,
and $\Sigma_k$ by maximizing the log-likelihood. Given $N$ i.i.d (Independent
Identically Distributed) data samples, derive the update formula for
each parameter. Note that in order to obtain an update rule for the M-step, we fix the responsibilities, i.e. $p(z_k^n|x_n)$, which we have already calculated in the E-step. [15 pts]}

\emph{Hint:} Use Lagrange multiplier for $\pi_k$ to apply
constraints on it.
\\Answer:
\begin{equation} \nonumber
\begin{split}p(z_k^i=k|x^i) &= \frac{p(z^i=k,x^i)}{\sum_{k=1..K} p(z^i=k,x^i)}\\
&=\frac{\pi_k \mathcal{N}(x^i|\mu_k,\Sigma_k)}{\sum_{k=1..K'}\pi_k \mathcal{N}(x^i|\mu_k,\Sigma_k)}\\
\end{split}
\end{equation}

\begin{equation} \nonumber
\begin{split}f(\theta) &= E_{q(z^1,z^2,..,z^m)}[\log\prod_{i=1}^m p(x^i,z^i|\theta)]\\
&= \sum_{i=1}^m E_{p(z^i|x^i,\theta^t)}[\log p(x^i,z^i|\theta)]\\
&= \sum_{i=1}^m E_{p(z^i|x^i,\theta^t)}[\log \pi_{z^i}\mathcal{N}(x^i|\mu_{z^i},\Sigma_{z^i})]\\
\end{split}
\end{equation}
Expanding the log of Gaussian, we get: $$log\mathcal{N}(x^i|\mu_{z^i},\Sigma_{z^i})$$

\begin{equation} \nonumber
\begin{split}f(\theta) &= \sum_{i=1}^m E_{p(z^i|x^i,\theta^t)}[log \pi_{z^i} - (x^i - \mu_{z^i})^T \Sigma_{z^i}(x^i - \mu_{z^i}) + log\Sigma_k + c]\\
&=\sum_{i=1}^m E_{p(z^i|x^i,\theta^t)}[log \pi_{z^i} - (x^i - \mu_{z^i})^T \Sigma_{z^i}(x^i - \mu_{z^i}) + log\Sigma_{z^i} + c]\\
&=\sum_{i=1}^m E_{p(z^i=k|x^i)}[log \pi_{k} - (x^i - \mu_{k})^T \Sigma_{k}(x^i - \mu_{k}) + log\Sigma_k + c]\\
f(\theta) &= \sum_{i=1}^m \sum_{k=1}^K E_{p(z^i=k|x^i)}[\log \pi_k - (x^i - \mu_k)^T \Sigma_k (x^i - \mu_k) + \log \Sigma_k + c]\\ 
\end{split}
\end{equation}
Hence, to find $\pi_k$, we apply the Lagrange multiplier, and use the information that $\sum_{i=1}^K\pi_k=1$ \\
Forming the Lagrangian, we obtain:
$$L = \sum_{i=1}^m \sum_{k=1}^K E_{p(z^i=k|x^i)}[\log \pi_k + terms] + \lambda(1 - \sum_{i=1}^K\pi_k)$$ 
\begin{equation}
J =  \sum_{i=1}^m \sum_{k=1}^K \tau_k^i[\log \pi_k + terms] + \lambda(1 - \sum_{i=1}^K\pi_k)\end{equation}

Taking partial derivative and equating to 0, we get:
$$\frac{\partial{L}}{\partial{\pi_k}} = \sum_{i=1}^m \frac{p(z^i=k|x^i)}{\pi_k} - \lambda = 0$$
$$\pi_k = \frac{1}{\lambda} \sum_{i=1}^m p(z^i=k|x^i)$$
$$\lambda = m$$
\newline

For the other two parts, we need to expand the middle term \\
 Expansion of squared Mahalanobis distance. \\
\begin{equation} \nonumber
\begin{split}
(x - \mu_k)^t \Sigma_{k}^{-1}(x - \mu_k)\\
=&x^t \Sigma_{k}^{-1}x -x^t\Sigma^{-1}u_k - u_{k}^t \Sigma^{-1}x + u_{k}^t\Sigma^{-1}u_{k} \\
=&x^t \Sigma^{-1}x -2(\Sigma^{-1}u_k)^t x + u_{k}^t\Sigma^{-1}u_{k}  
 \end{split}
\end{equation}
here the term $x^t \Sigma^{-1}x $ is common in all and thus gets removed,  thus we get the following equation (note we ignore $\lambda$ as it was just dependent on $\pi_k$)
 
\begin{equation}f(\theta) = \sum_{i=1}^m \sum_{k=1}^K \tau_k^i[\log \pi_k - x^t \Sigma^{-1}_kx -2(\Sigma^{-1}_ku_k)^t x + u_{k}^t\Sigma^{-1}_ku_{k}  + \log \Sigma_k + c]\end{equation}

Now taking derivative wrt $\mu_k$ , we get

\begin{equation} \nonumber
\begin{split}
\frac{\partial{L}}{\partial{\mu_k}} =& \sum_{i=1}^m\tau_k^i( -2 \Sigma^{-1}_kx + 2\Sigma^{-1}_k\mu_k)=0 \\
\sum_{i=1}^m\tau_k^i2 x =&  \sum_{i=1}^m\tau_k^i(2\mu_k) \\
 \end{split}
\end{equation}
Here we removed $\Sigma^{-1}_k$ as it was independent of the number of points. Similarly, we in the second summation , we can right $\mu_k$ outside the summation
\begin{equation} \nonumber
\begin{split}
\frac{\partial{L}}{\partial{\mu_k}} =& \sum_{i=1}^m\tau_k^i( -2 \Sigma^{-1}_kx + 2\Sigma^{-1}_k\mu_k)=0 \\
\sum_{i=1}^m\tau_k^ix =&  \mu_k \sum_{i=1}^m\tau_k^i \\
\mu_k  =& \frac{\sum_{i=1}^m\tau_k^ix}{\sum_{i=1}^m\tau_k^i} \\
 \end{split}
\end{equation}

Similarly as in the last equation, we may right the last equation as 

\begin{equation}f(\theta) = \sum_{i=1}^m \sum_{k=1}^K \tau_k^i[\log \pi_k - x^t \Sigma^{-1}_kx -2(\Sigma^{-1}_ku_k)^t x + u_{k}^t\Sigma^{-1}_ku_{k}  - \log \Sigma^{-1}_k + c]\end{equation}

Taking derivative wrt to $\Sigma_k^{-1}$ , we get
\begin{equation} \nonumber
\begin{split}
\frac{\partial{L}}{\partial{\Sigma_k^{-1}}} =& \sum_{i=1}^m\tau_k^i(  x^t x -2(\mu_k)^t x + \mu_{k}^t\mu_{k}- \frac{1}{\Sigma_k^{-1}})=0 \\
\sum_{i=1}^m\tau_k^i((x-\mu_k)^t(x-\mu_k)) =&  \Sigma_k \sum_{i=1}^m\tau_k^i \\
 \Sigma_k =& \frac{\sum_{i=1}^m\tau_k^i((x-\mu_k)^t(x-\mu_k)) }{ \sum_{i=1}^m\tau_k^i } \\
 \end{split}
\end{equation}

\subsubsection*{(d) EM and K-Means [10 pts]}
K-means can be viewed as a particular limit of EM for Gaussian
mixture. Considering a mixture model in which all components have
covariance $\epsilon I$, show that in the limit $\epsilon\to 0$,
maximizing the expected complete data log-likelihood for this model
is equivalent to minimizing objective function in K-means:

\begin{equation}
J = \sum_{n=1}^N\sum_{k=1}^K\gamma_{nk}\|x_n-\mu_k\|^2,\nonumber
\end{equation}
where $\gamma_{nk} = 1$ if $x_n$ belongs to the $k$-th cluster and $\gamma_{nk} = 0$ otherwise.

In E step we calculate $\gamma (k)$ , which is basically p(k/x). From the above expressions we got :


\begin{equation}
p(x) = \frac{\pi_k\mathcal{N}(x|\mu_k,
\Sigma_k)}{\sum_{j=1}^K \pi_j \mathcal{N}(x|\mu_j,
\Sigma_j)}.\nonumber
\end{equation}
\begin{equation}
p(k/x) = \frac{\pi_k\exp
\left( - \frac{(x - \mu_k)^t \Sigma^{-1}(x - \mu_k)}{2} \right)}{\sum_{j=1}^K \pi_j \exp
\left( - \frac{(x - \mu_j)^t \Sigma^{-1}(x - \mu_j)}{2} \right)}.\nonumber
\end{equation}


Inverse of Covariance Matrix 
 \begin{equation} \nonumber
\begin{split}
\Sigma_{k}^{-1}= (1/ \epsilon )* I
 \end{split}
\end{equation}, So its only effect is to scale vector product by $1/ \epsilon$
Thus,
\begin{equation} \nonumber
\begin{split}
p(k/x) =& \frac{\pi_k\exp.
\left( - \frac{(x - \mu_k)^t (x - \mu_k) }{2\epsilon} \right)}{\sum_{j=1}^K \pi_j \exp
\left( - \frac{(x - \mu_j)^t (x - \mu_j)}{2\epsilon} \right)}\\
=&\frac{\pi_k\exp 
\left( - \frac{||(x - \mu_k)||^2 }{2\epsilon} \right)}{\sum_{j=1}^K \pi_j \exp
\left( - \frac{||(x - \mu_j)||^2}{2\epsilon} \right)}.
\end{split}
\end{equation}
Now , in order to get the limiting value when $\epsilon =0$ , let us divide numerator and denominator by \\
$\pi_k\exp
\left( - \frac{(x - \mu_k)^t (x - \mu_k) }{2\epsilon} \right)$ \\
Thus we get the equation of the form :
\begin{equation} \nonumber
\begin{split}
p(k/x) =& \frac{1}{\frac{\sum_{j=1}^K \pi_j \exp
\left( - \frac{||(x - \mu_j)||^2}{2\epsilon}  \right)}{ \pi_k\exp.
\left( - \frac{||(x - \mu_k) ||^2 }{2\epsilon} \right)}} \\
=& \frac{1}{1 +\sum_{j=1, j\neq k}^K (\pi_j /\pi_k) \exp
\left( - \frac{||(x - \mu_j)||^2-||(x - \mu_k)||^2}{2\epsilon }   \right)} \\
\end{split}
\end{equation}
Now for the above equation, for the k with the maximum posterior probability  $||(x - \mu_k)||^2$ is will smallest as it will be closest to the point x , thus when at limit $\epsilon  = 0$ , then all the vaues in summation will be zero as  $||(x - \mu_j)||^2-||(x - \mu_k)||^2$ will be greater than zero for all j's, thus negative of exponential of that number would tend to 0. \\However , for all the other values of k , there will be at least one value in summation such that $||(x - \mu_j)||^2-||(x - \mu_k)||^2$ will be less than zero, thus its negotiation will be less than zero and so at limit would blow to infinity(denominator). Thus p(k/x) for all the others will be zero. This is exactly what happens in K means as $\gamma (k)$ would be 1 for only one cluster and zero for the rest. Thus it exactly corresponds to the equation given above.
\subsubsection*{(e) General setting [10 pts]}

Consider a mixture of distribution of the form
\begin{equation}
P(x) = \sum_{k=1}^K \pi_k p(x|k)\nonumber
\end{equation}
where the elements of $x$ could be discrete or continuous or a
combination of these. Express the mean and covariance of the mixture
distribution using the mean $\mu_k$ and covariance $\Sigma_k$ of
each component distribution $p(x|k)$.



Now, the expectation of the mixture distribution can be given by:

\begin{equation} \nonumber
\begin{split}
E[x] &= \sum_{x} x P(x) \\
&= \sum_{x} x \sum_{k=1}^K \pi_k p(x|k) \\
&= \sum_{k=1}^K \pi_k \sum_{x} x p(x|k) \\
&= \sum_{k=1}^K \pi_k E[x|k] \\
E[x] &= \sum_{k=1}^K \pi_k \mu_k
\end{split}
\end{equation} \\
Note that above, for Expection of x given k is  basically the mean for the distribution k cluster.\\
For finding the covariance, the following general relation can be used:

\begin{equation} \nonumber
\begin{split}
\Sigma = E[x x^{T}] - E[x] E[x]^{T}
\end{split}
\end{equation} \\

Therefore, we get:

\begin{equation} \nonumber
\begin{split}
\Sigma &= E[x x^{T}] - E[x] E[x]^{T} \\
\Sigma &= \sum_{k=1}^K \pi_k E_k [x x^{T}] - E[x] E[x]^{T}
\end{split}
\end{equation} \\

Therefore, we get:

\begin{equation} \nonumber
\begin{split}
\Sigma &= \sum_{k=1}^K \pi_k E_k [x x^{T}] - E[x] E[x]^{T} \\
\Sigma &= \sum_{k=1}^K \pi_k (\Sigma_k + \mu_k \mu_k^{T}) - E[x] E[x]^{T} 
\end{split}
\end{equation} 


\vspace{1cm}


\iffalse
\section{Mixture of Bernoulli Distributions}
In this problem, we are going to derive the EM algorithm for a
different mixture model, mixture of Bernoulli distributions. We use
the same 1-of-$K$ coding scheme, which we introduced in Problem 1.

First, the definition of Bernoulli distribution is given by
\begin{equation}
p(x|\mu) = \prod_{i=1}^D \mu_i^{x_i} (1 - \mu_i)^{1 - x_i}.\nonumber
\end{equation}
A mixture of $K$ Bernoulli distribution components can be expressed
as
\begin{equation}
p(x|\mu, \pi) = \sum_{k=1}^K \pi_k \prod_{i=1}^D \mu_i^{x_i} (1 -
\mu_i)^{1 - x_i},\nonumber
\end{equation}
where $\pi_k$ represents the probability for a data point to belong
to the $k$th component. As it is probability, $\sum_k \pi_k = 1$.

\subsubsection*{(a) Rewrite the formula $p(X, Z|\mu, \pi)$
(now we have $Z$ in the arguments) for mixture of Bernoulli
distributions with 1-of-$K$ coding for latent variables. [5 pts]}

\subsubsection*{(b) Suppose we have $N$ i.i.d (Independent Identically
Distributed) data samples from the distribution in (a). Write down
the log-likelihood function $\ln p(X, Z|\mu, \pi)$. [5 pts]}

\subsubsection*{(c) In E-Step, we re-evaluate responsibility of each
component using Bayes' theorem. Write down the update formula for
$\mathbb{E}[z_k^n]$. [5 pts]}

\subsubsection*{(d) In M-Step, we re-estimate parameters $\mu$ and $\pi$
by maximizing the expected complete-data log-likelihood with respect
to them. Derive the update formula for both $\mu_k$ and $\pi_k$. [10
pts]}

\vspace{1cm} \fi


\section{Density Estimation}

Consider a histogram-like density model in which the space $x$ is
divided into fixed regions for which density $p(x)$ takes constant
value $h_i$ over $i$th region, and that the volume of region $i$ in
denoted as $\Delta_i$. Suppose we have a set of $N$ observations of
$x$ such that $n_i$ of these observations fall in regions $i$.

\subsubsection*{(a) What is the log-likelihood function? [8 pts]}


Probability is basically mass which can also be represented by density and volume.
Now, over a region $i$ where volume of the region is given by $\Delta_i$:

$$
P(x) = h_i \Delta_i\hspace{10mm} over\ i^{th}\ region
$$

And we are given a set of $N$ observations of $x$ such that $n_i$ of these observations fall in regions $i$. Therefore, the likelihood is given by:

$$
P(x|data) = \prod_{i} (h_i \Delta_i)^{n_i}
$$

Now, the log-likelihood is given by:

$$
\log P(x|data) = \log \prod_{i} (h_i \Delta_i)^{n_i} \\
= \sum_{i} n_i \log (h_i \Delta_i)
$$

\subsubsection*{(b) Derive an expression for the maximum likelihood estimator for {$h_i$}. [10 pts]}

\emph{Hint:} This is a constrained optimization problem. Remember that $p(x)$ must integrate to unity. Since
$p(x)$ has constant value $h_i$ over region $i$, which has volume
$\Delta_i$. The normalization constraint is $\sum_i h_i\Delta_i =
1$. Use Lagrange multiplier by adding
$\lambda\left(\sum_ih_i\Delta_i-1\right)$ to your objective function.

As this is a constrained optimization problem, we need to use Lagrange multiplier. So, the following equation is:

\begin{equation} \nonumber
\begin{split}
\log P(x|data) &= \sum_{i} n_i \log (h_i \Delta_i) + \lambda\left(\sum_ih_i\Delta_i-1\right)
\end{split}
\end{equation}

Now differentiating the log-likelihood with respect to $h_i$, we get:

\begin{equation} \nonumber
\begin{split}
\frac{\partial \log P(x|data)}{\partial h_i} &= \frac{\partial}{\partial h_i} \left\{ \sum_{i} n_i \log (h_i \Delta_i) + \lambda\left(\sum_ih_i\Delta_i-1\right) \right\} \\
&= n_i \frac{1}{h_i \Delta_i} \Delta_i + \lambda \Delta_i \\
&= \frac{n_i}{h_i} + \lambda \Delta_i 
\end{split}
\end{equation}

Equating this to $0$, we get:

\begin{equation} \nonumber
\begin{split}
h_i &= - \frac{n_i}{\lambda \Delta_i} 
\end{split}
\end{equation}

Substituting this value back in the constraint $\sum_i h_i\Delta_i =1$, we get:

\begin{equation} 
\begin{split}
\sum_i \left( - \frac{n_i}{\lambda \Delta_i} \right) \Delta_i &= 1 \\
\lambda &= - \sum_i n_i \\
\lambda &= -N
\end{split}
\end{equation}

Therefore, we get $h_i$:

\begin{equation} \nonumber
\begin{split}
%\frac{n_i}{h_i} + \lambda \Delta_i &= 0 \\
h_i &= - \frac{n_i}{\lambda \Delta_i} \\
h_i &= \frac{n_i}{N \Delta_i}
\end{split}
\end{equation}


\subsubsection*{(c) Mark $T$ if it is always true, and $F$ otherwise. Briefly explain why. [12 pts]}

\begin{itemize}
  \item Non-parametric density estimation usually does not have parameters. 
  
  FALSE. Because Non-parametric density estimation has parameters dependent on the number of data points. The difference between parametric model and non-parametric model is that the former has a fixed number of parameters, while the latter grows the number of parameters with the amount of training data. As an Example, Histograms make no assumptions about the probability distributions of the variables being assessed.
    
  \item The Epanechnikov kernel is the optimal kernel function for all data.
  
  TRUE. Epanechnikov kernel is the optimal kernel since it minimizes the AMISE (Asymptotic Mean Integrated Squared Error) and hence is the optimal kernel function for all data. Kernel efficiency is measured in comparison to the Epanechnikov Kernel. This Kernel function minimizes the risk as well, hence it can be considered the optimal kernel.  
  
  \item Histogram is an efficient way to estimate density for high-dimensional data.
  
  FALSE. Histograms are not an efficient since in case there is high-dimensional data, there are too many bins and it gets in the problem of sparse bins where weight of bins is sufficient in only a few number of bins. For instance, if bin size is $\frac{1}{n}$ and there are m i.i.d samples of data and the data is split into bins = $n^d$ if $n^d > m$  then most bins are empty, hence it is not a good utilization of memory or space as well. \\
  Another reason is that statistically histograms are not better than other models like KDE. The integrated risk is much more in case of histogram than other models like the KDE. Integrated risk converges at the rate of $O(m^{-\frac{2}{3}})$ in case of histograms where as for the KDE it converges at the rate of $O(m^{-\frac{4}{5}})$. This impact is even more significant when the data is in high dimensions.
  
  
  \item Parametric density estimation assumes the shape of probability density.

  TRUE. In case of the parametric approach, we assume that the density function takes on a particular form, and where its corresponding parameters are estimated by data. When the parametric model consists of absolutely continuous distributions, it is often specified in terms of corresponding probability density functions.
  
  \end{itemize}


\vspace{1cm}


\section{Information Theory}
In the lecture you became familiar with the concept of entropy for one random variable and mutual information. For a pair of discrete random variables $X$ and $Y$ with the joint distribution $p(x,y)$, the \emph{joint entropy} $H(X,Y)$ is defined as
\begin{equation}
H(X,Y)=-\sum_{x\in X}\sum_{y\in Y}{p(x,y)\log p(x,y)}
\end{equation}
which can also be expressed as
\begin{equation}
H(X,Y)=-\mathbb{E}[\log p(X,Y)]
\end{equation}
Let $X$ and $Y$ take on values $x_1,x_2,...,x_r$ and $y_1,y_2,...,y_s$ respectively. Let Z also be a discrete random variable and $Z=X+Y$.
\newline
\newline
\textbf{(a)} Prove that $H(X,Y)\leq H(X)+H(Y)$ [4 pts] \newline

We can write the joint probability of $x$ and $y$ as:

\begin{equation} \nonumber
\begin{split}
p(x,y) &= p(y|x) p(x)  
\end{split}
\end{equation}

Substituting this in $H(x,y)$, we get:

\begin{equation} \nonumber
\begin{split}
H(X,Y) &= - \sum_{x \in X}\sum_{y \in Y}{p(x,y)\log p(x,y)} \\
&= - \sum_{x \in X}\sum_{y \in Y}{p(x,y)\log p(y|x) p(x)} \\
&= - \sum_{x \in X}\sum_{y \in Y}{p(x,y)\log p(y|x)}  - \sum_{x \in X}\log p(x)\sum_{y \in Y}{p(x|y) p(y)} \\
&= - \sum_{x \in X}\sum_{y \in Y}{p(x,y)\log p(y|x)}  - \sum_{x \in X} p(x) \log p(x) \\
&= - \sum_{x \in X}\sum_{y \in Y}{p(x,y)\log p(y|x)}  + H(X) \\
&= H(Y|X) + H(X)
\end{split}
\end{equation}

Then, we get:

\begin{equation} \nonumber
\begin{split}
H(Y) - H(Y|X) &= - \sum_{x \in X}\sum_{y \in Y}{p(x,y)\log p(y)} + \sum_{x \in X}\sum_{y \in Y}{p(x,y)\log p(y|x)} \\
&= \sum_{x \in X}\sum_{y \in Y}{p(x,y) \log \frac{p(y|x)}{p(y)}} \\
&\geq 0
\end{split}
\end{equation}

Therefore, we get:

\begin{equation} \nonumber
\begin{split}
H(Y) - H(Y|X) &\geq 0 \\
H(Y|X) &\leq H(Y)
\end{split}
\end{equation}

Thus, using this inequality, we get that:

\begin{equation} \nonumber
\begin{split}
H(X,Y) &= H(Y|X) + H(X) \\
&\leq H(X) + H(Y)
\end{split}
\end{equation}

\textbf{(b)} If $X$ and $Y$ are independent, i.e. $P(X,Y)=P(X)P(Y)$, then $H(X,Y)=H(X)+H(Y)$ [4 pts] \newline\newline
\\Answer: 
\begin{equation*}
H(X,Y) = -\sum_{x\in X}\sum_{y\in Y}{p(x,y)\log p(x,y)}
\end{equation*}
Since X and Y  are independent: $$P(X,Y) = P(X)P(Y)$$, Thus we get
\begin{equation}
H(X,Y) = -\sum_{x\in X}\sum_{y\in Y}{p(x)p(y)\log (p(x)p(y))}
\end{equation}
Using log(p(x)p(y)) as log(p(x)) + log(p(y), we get:
\begin{equation*}
\begin{split}
=& -\sum_{x\in X}\sum_{y\in Y}{p(x)p(y)\log p(x)} -\sum_{x\in X}\sum_{y\in Y}{p(x)p(y)\log p(y)} \\
=& -\sum_{y\in Y}p(y)\sum_{x\in X}{p(x)\log p(x)} -\sum_{x\in X}p(x)\sum_{y\in Y}{p(y)\log p(y)}
\end{split}
\end{equation*}
Summation of x and y in their space is one.Thus, p(y) and p(x) in the respective terms above clearly sum up to 1. 
\begin{equation*}
= -1 * \sum_{x\in X}{p(x)\log p(x)} - 1 * \sum_{y\in Y}{p(y)\log p(y)}
\end{equation*}
\begin{equation*}
= -\sum_{x\in X}{p(x)\log p(x)} - \sum_{y\in Y}{p(y)\log p(y)}
\end{equation*}
Since,
\begin{equation*}
-\sum_{x\in X}{p(x)\log p(x)} = H(X)
\end{equation*}

\begin{equation}
-\sum_{y\in Y}{p(y)\log p(y)} = H(Y)
\end{equation}
Thus , we get
\begin{equation}
=H(X) + H(Y)
\end{equation}

Hence, proved that 
\begin{equation*}
H(X,Y) = H(X) + H(Y)
\end{equation*}
\newline
\textbf{(c)} Show that $I(X;Y)=H(X)+H(Y)-H(X,Y)$. [4 pts] \newline\newline
\\Answer:
Mutual information of two discrete random variables X and Y is defined as  :
\begin{equation}
\begin{split}
I(X;Y) =& \sum_{x\in X}\sum_{y\in Y}{p(x,y)\log (\frac{p(x,y)}{p(x)p(y)})} \\
=& \sum_{x\in X}\sum_{y\in Y}{p(x,y)\log(p(x,y))} - \sum_{x\in X}\sum_{y\in Y}{p(x,y)\log(p(x))} - \sum_{x\in X}\sum_{y\in Y}{p(x,y)\log(p(y))} \\
=& - H(X,Y) - \sum_{x\in X}\log(p(x))\sum_{y\in Y}{p(x,y)} - \sum_{y\in Y}\log(p(y))\sum_{x\in X}{p(x,y)} 
\end{split}
\end{equation}
In the above equation we took $\log(p(y))$ and $\log(p(x))$ out of summation as their are constant wrt to other summation.\\Note that second terms are  marginalization equations x and y respectively, $$ \sum_{y\in Y}{p(x,y)}  = p(x) $$, Thus we can say that: \\
 
$$\sum_{x\in X}\log(p(x))\sum_{y\in Y}{p(x,y)} = \sum_{x\in X}p(x)\log(p(x))$$
and
$$\sum_{y\in Y}\log(p(y))\sum_{x\in X}{p(x,y)} = \sum_{y\in Y}p(y)\log(p(y))$$
Hence, the earlier equation can be written as:
\begin{equation*}
\begin{split}
=& - H(X,Y) - \sum_{x\in X}p(x)\log(p(x)) - \sum_{y\in Y}p(y)\log(p(y)) \\
=& - H(X,Y) + H(X) + H(Y) \\
\end{split}
\end{equation*}
\newline
\textbf{(d)} Show that $H(Z|X)=H(Y|X)$. Argue that when $X,Y$ are independent, then $H(X) \leq H(Z)$ and $H(Y) \leq H(Z)$. Therefore, the addition of \emph{independent} random variables add uncertainty. [4 pts] \newline\newline
Answer:
Condition entropy over Random variables can be expressed as summation of individual conditional entropies in space. Thus,
\begin{equation*}
\begin{split}
H(Z|X) =& \sum_{x\in X}P(x)H(Z|X=x)\\
=& -\sum_{x\in X}P(x)\sum_{z\in Z}P(Z=z|X=x)logP(Z=z|X=x) \\
=& -\sum_{x\in X}P(x)\sum_{y\in Y}P(Y=z-x|X=x)logP(Y=z-x|X=x) \\
=& -\sum_{x\in X}P(x)H(Y|X=x) \\
=& H(Y|X) \\
\end{split}
\end{equation*}
Hence, proved that:
\begin{equation*}
H(Z|X) = H(Y|X)
\end{equation*}
\newline

\begin{equation*}
\begin{split}
H(Y|X)=& \sum_{x\in X}p(x)H(Y|X=x) \\
=& -\sum_{x\in X}p(x)\sum_{y\in Y}p(y|x)\log p(y|x) \\
\end{split}
\end{equation*}
Since X and Y are independent, we can 
\begin{equation*}
= -\sum_{x\in X}p(x)\sum_{y\in Y}p(y)\log p(y)
\end{equation*}
First summation across the domain would be 1. Hence, we get:
\begin{equation*}
= -\sum_{y\in Y}p(y)\log p(y)= H(Y)
\end{equation*}
Hence, we have proved that:
\begin{equation*}
H(Y|X) = H(Y)
\end{equation*}
Since Z is a function of X and Y , so knowing X or conditioning would reduce entropy, thus we get: 
\begin{equation*}
H(Z) \geq H(Z|X) = H(Y|X) = H(Y)	
\end{equation*}
\begin{equation*}
H(Z) \geq H(Z)
\end{equation*}
Rewriting, we get:
\begin{equation*}
Hence, H(Y) \leq H(Z)
\end{equation*}
Similarly, we can also prove that 
\begin{equation*}
H(X) \leq H(Z)
\end{equation*}
Hence, proved.
\newline\newline
\textbf{(e)} Under what conditions does $H(Z)=H(X)+H(Y)$. [4 pts] 
\newline\newline
Answer: \\H(Z) can be represented as H(X+Y). We have already proved that 
 If $X$ and $Y$ are independent, i.e. $P(X,Y)=P(X)P(Y)$, then $H(X,Y)=H(X)+H(Y)$ \\
Also, we can see here  that $H(Z) <=H(X,Y)$ , because $H(Z)$ is a function for X,Y and for a function there may be multiple ways to get a particular value. Lets say Z=5, then we can get 5 from $(1,4)$ or (2,3). If we want the equality to hold, that is
\begin{equation}
H(X+Y)=H(X,Y)
\end{equation}
if the addition is invertible, that is, if there are unique $X$ and $Y$ for any $X+Y$; this would happen, for example, if ${\cal X}=\{1,2\}$ and ${\cal Y}=\{1,3\}$ since the possible values of the sum are 2, 3, 4 and 5 and each corresponds to a different choice of $X$ and $Y$; however, if  ${\cal X}=\{1,2\}$ and ${\cal Y}=\{1,2\}$ then $X=1$, $Y=2$ and $X=2$, $Y=1$ both give $X+Y=3$. Now
\begin{equation}
H(X,Y)=H(X)+H(Y|X)
\end{equation}
Thus, the conditions for independence are that function for $X+Y$ should be bijective and X and Y should be independent for the condition to hold true.
\newline
\newline
\newline
\vspace{1cm}


\vspace{1cm}

\section{Bayes Classifier}
\subsection{\label{bcwglf} Bayes Classifier With General Loss Function}
In class, we talked about the popular 0-1 loss function in which $L(a,b) = 1$ for $a\neq b$ and 0 otherwise, which means all wrong predictions cause equal loss. Yet, in many other cases including cancer detection, the asymmetric loss is often preferred (misdiagnosing cancer as no-cancer is much worse). In this problem, we assume to have such an asymmetric loss function where $L(a,a) = L(b,b) = 0$ and $L(a,b)=p, L(b,a) = q, p\neq q$. Write down
the the Bayes classifier $f:X\rightarrow Y$ for binary
classification $Y\in\{-1,+1\}$. Simplify the classification rule as much as you can. [20
pts] \newline \\[.20cm]
The optimal solution is the one which minimizes the loss function. However,
the loss function depends on the true class, which is unknown. For a given input
vector x, our uncertainty in the true class is expressed through the joint probability
distribution p(x, Ck) and so we seek instead to minimize the average loss, where the
average is computed with respect to this distribution, which is given by
\begin{equation}
E[L] =  \sum_k  \sum_j \int_{R_j} L_{kj} p(x, C_k) \nonumber
\end{equation}
Each x can be assigned independently to one of the decision regions $R_j$ . Our goal
is to choose the regions $R_j$ in order to minimize the expected loss , which
implies that for each x we should minimize
\begin{equation}
\sum_k L_{kj}p(x, C_k).\nonumber
\end{equation}

  Thus , we can do the following

\begin{equation} \nonumber
\begin{split}
 Min \sum_k L_{kj}p(x, C_k) \\
 => Min \sum_k L_{kj}p(C_k | x) *p(x)
 => Min \sum_k L_{kj}p(C_k | x) 
  \end{split}
\end{equation}
Note -- We can ignore p(x) as summation is across k thus  it is a same constant in all terms\\
Let us denote this term by risk so for a particular class k

\begin{equation} \nonumber
\begin{split}
 R(k_1) &=  \sum_k L_{kk_1}p(C_k | x) \\ \
 \end{split}
\end{equation}
Thus Bayesian Rule for the Binary class will be 

\begin{equation} \nonumber
\begin{split}
 R(k_1) < R(k_2) \\ 
 \end{split}
\end{equation}
where
\begin{equation} \nonumber
\begin{split}
 R(k_1) = L_{k_1k_1}p(C_{k_1} | x)  + L_{k_2k_1}p(C_{k_2} | x) \\ 
 R(k_2) = L_{k_2k_2}p(C_{k_2} | x)  + L_{k_1k_2}p(C_{k_1}| x) \\ 
\end{split}
\end{equation}
We are given that L(a,a) and L(b,b) are zero. Thus, our decision rule becomes
\begin{equation} \nonumber
\begin{split}
 Min ( L(b,a)p(a|x), L(a,b)p(b|x)) \\
 Min (p*p(a|x)) , q*p(b|x)
 \end{split}
\end{equation}
We can convert this into class condition using
\begin{equation} \nonumber
\begin{split}
 p(a|x) = p(x|a)p(a)/p(x)
 \end{split}
\end{equation}
Here again we may ignore p(x).Thus our equation becomes and this is out decision boundary for the given case.

\begin{equation} \nonumber
\begin{split}
  Min ((p*p(x|a)*p(a)) ,( q*p(x|b)*p(b)))
 \end{split}
\end{equation}
\subsection{Gaussian Class Conditional distribution}
\paragraph{(a)} Suppose the class conditional distribution is a Gaussian.
Based on the general loss function in problem \ref{bcwglf}, write
the Bayes classifier as $f(X) = \text{sign}(h(X))$ and simplify $h$
as much as possible. What is the geometric shape of the decision
boundary? [10 pts]

\paragraph{(b)} Repeat (a) but assume the two Gaussians have identical
covariance matrices. What is the geometric shape of the decision
boundary? [10 pts]

\paragraph{(c)} Repeat (a) but assume now that the two Gaussians have covariance
matrix which is equal to the identity matrix. What is the geometric
shape of the decision boundary? [10 pts]\\

\paragraph{Answer}

For all the 3 parts , we will first define a log for the earlier decision boundary for better understanding of the problem
\begin{equation} \nonumber
\begin{split}
g_k(x)  =  \ln(loss_{k}) + \ln(p(x|k) + \ln(p(a))
 \end{split}
\end{equation},

Here we have Normal Distribution for a class
\begin{equation}
p(x) = \frac{1}{ (2\pi)^{d/2} \Sigma^{1/2}} \exp
\left( - \frac{(x - \mu)^t \Sigma^{-1}(x - \mu)}{2} \right).\nonumber
\end{equation}
Thus ,
\begin{equation} \nonumber
\begin{split}
g_k(x)  =  \ln(loss_{k}) + \frac{(x - \mu_k)^t \Sigma_{k}^{-1}(x - \mu_k)}{2} -d/2 \ln(2\pi) -1/2 \ln\Sigma_k+ \ln(p(k))
 \end{split}
\end{equation}
This is our basic equation which will be used across all the three parts . 


\paragraph{Case a : Arbitary}
Now for the first case, we may ignore $d/2 \ln(2\pi) $ as it is same across all the classes. \\[.20cm]
Expansion of squared Mahalanobis distance. \\
\begin{equation} \nonumber
\begin{split}
(x - \mu_k)^t \Sigma_{k}^{-1}(x - \mu_k)\\
=&x^t \Sigma_{k}^{-1}x -x^t\Sigma_{k}^{-1}u_k - u_{k}^t \Sigma_{k}^{-1}x + u_{k}^t\Sigma_{k}^{-1}u_{k} \\
=&x^t \Sigma_{k}^{-1}x -2(\Sigma_{k}^{-1}u_k)^t x + u_{k}^t\Sigma_{k}^{-1}u_{k}  
 \end{split}
\end{equation}
the last step comes from symmetry of the 
covariance matrix and thus its inverse:

Thus putting this value in equation we get a quadratic equation:
\begin{equation} \nonumber
\begin{split}
x^t (\Sigma_{k}^{-1}/2)x -((\Sigma_{k}^{-1}u_k)^t )x + u_{k}^t\Sigma_{k}^{-1}u_{k}  + \ln(loss_{k}) -1/2 \ln\Sigma_k+ \ln(p(k))
 \end{split}
\end{equation}
The above is a quadratic equation in terms of the variable x and Decision Boundary which basically corresponds to 
$$ g_1(x)  =g_2(x) $$ can thus  are hyperquadratics, and they can assume any of the general forms: 
hyperquadrics: 
can be hyperplanes, 
\begin{enumerate}
\item hyperplane pairs, 
\item hyperspheres,
\item hyperellipsoids,
\item hyperparabaloids, 
\item hyperhyperparabaloids .
\end{enumerate}
\paragraph{Case b : Identical Covariance}
\begin{equation} \nonumber
\begin{split}
g_k(x)  =  \ln(loss_{k}) + \frac{(x - \mu_k)^t \Sigma_{k}^{-1}(x - \mu_k)}{2} -d/2 \ln(2\pi) -1/2 \ln\Sigma_k+ \ln(p(k))
\end{split}
\end{equation}

 For identical covariance $\Sigma_{k} =\Sigma $  for all k's. \\Thus we can remove terms 
 $ -d/2 \ln(2\pi) -1/2 \ln\Sigma_k $, so our equation now is : 
 \begin{equation} \nonumber
\begin{split}
g_k(x)  =  \ln(loss_{k}) + \frac{(x - \mu_k)^t \Sigma_{k}^{-1}(x - \mu_k)}{2}+ \ln(p(k))
 \end{split}
\end{equation}


Expansion Expansion of squared Mahalanobis distance. \\
\begin{equation} \nonumber
\begin{split}
(x - \mu_k)^t \Sigma_{k}^{-1}(x - \mu_k)\\
=&x^t \Sigma_{k}^{-1}x -x^t\Sigma^{-1}u_k - u_{k}^t \Sigma^{-1}x + u_{k}^t\Sigma^{-1}u_{k} \\
=&x^t \Sigma^{-1}x -2(\Sigma^{-1}u_k)^t x + u_{k}^t\Sigma^{-1}u_{k}  
 \end{split}
\end{equation}
here the term $$x^t \Sigma^{-1}x $$ is common in all and thus gets removed,  thus we get the following equation
\begin{equation} \nonumber
\begin{split}
g_i(x)=((\Sigma^{-1}u_k)^t )x + u_{k}^t\Sigma^{-1}u_{k}  + \ln(loss_{k}) -1/2 \ln\Sigma+ \ln(p(k)) \end{split}
\end{equation}
Decision Boundary which basically corresponds to 
$$ g_1(x)  -g_2(x) = 0 $$ , 
So we get the following equation:
\begin{equation} \nonumber
\begin{split}
((\Sigma^{-1}u_1 - \Sigma^{-1}u_0)^t )x =\frac{-(u_{1}\Sigma^{-1}u_{1} -u_{0}\Sigma^{-1}u_{0})}{2}  + \ln\frac{p*p(a)}{q*p(b)}  \end{split}
\end{equation}
By Multiplying  
\begin{equation} \nonumber
\begin{split}
\frac{(\mu_1 - \mu_0)^t \Sigma^{-1}(\mu_1 - \mu_0)}{(\mu_1 - \mu_0)^t \Sigma^{-1}(\mu_1 - \mu_0)}
\end{split}
\end{equation} 
to $\frac{p*p(a)}{q*p(b)}  $ and manipulating the common variables in the equation, we get the following equation

\begin{equation} \nonumber
\begin{split}
((\Sigma^{-1}u_1 - \Sigma^{-1}u_0)^t )x =(\Sigma^{-1}u_1 - \Sigma^{-1}u_0)^t )( \frac{\mu_1+\mu_0}{2} +(\mu_1-\mu_0)*\frac{ \ln\frac{p*p(a)}{q*p(b)} }{(\mu_1 - \mu_0)^t \Sigma^{-1}(\mu_1 - \mu_0)}) \end{split}
\end{equation}

Thus can be written as norm form of equation
\begin{equation} \nonumber
\begin{split}
w^t(x-x_0) = 0\end{split}
\end{equation}
where $x_0$ is 
\begin{equation} \nonumber
\begin{split}
( \frac{\mu_1+\mu_0}{2} +(\mu_1-\mu_0)*\frac{ \ln\frac{p*p(a)}{q*p(b)} }{(\mu_1 - \mu_0)^t \Sigma^{-1}(\mu_1 - \mu_0)}) \end{split}
\end{equation}
Because the equations are linear, the resulting decision boundaries are  hyperplanes. \\
Thus , we will have a straight line as the boundary between the two.
A line through the point $x_0$ defines this decision boundary between 1 and 0. If the product of loss(i,j) prior probabilities (example p*p(a)) are equal then x0 is halfway between the means. If the products are not equal, the optimal boundary hyperplane is shifted away from the more likely mean .The decision boundary is in the direction orthogonal defined by 
 
\begin{equation} \nonumber
\begin{split}
(\Sigma^{-1}u_1 - \Sigma^{-1}u_0)^t 
\end{split}
\end{equation}

Thus, the boundary line will be tilted depending on how the 2 features covary and their respective variances.
\paragraph{Case c : Identidy Covariance $\sigma^2 I$}
\begin{equation} \nonumber
\begin{split}
g_k(x)  =  \ln(loss_{k}) + \frac{(x - \mu_k)^t \Sigma_{k}^{-1}(x - \mu_k)}{2} -d/2 \ln(2\pi) -1/2 \ln\Sigma_k+ \ln(p(k))
\end{split}
\end{equation}

We can remove terms 
 $ -d/2 \ln(2\pi) -1/2 \ln\Sigma_k $, so our equation now is : 
 \begin{equation} \nonumber
\begin{split}
g_k(x)  =  \ln(loss_{k}) + \frac{(x - \mu_k)^t \Sigma_{k}^{-1}(x - \mu_k)}{2}+ \ln(p(k))
 \end{split}
\end{equation}
Inverse of Covariance Matrix 
 \begin{equation} \nonumber
\begin{split}
\Sigma_{k}^{-1}= 1/ \sigma^2 * I
 \end{split}
\end{equation}, So its only effect is to scale vector product by $1/ \sigma^2$
Thus,
 \begin{equation} 
\begin{split}
g_k(x)  &=  \ln(loss_{k}) + \frac{(x - \mu_k)^t (x - \mu_k)}{2 \sigma^2}+ \ln(p(k))\\
&= \frac{x^tx -2\mu_{k}^tx +\mu_{k}^t\mu_{k}}{2 \sigma^2} +  \ln(loss_{k}) + \ln(p(k))\\
 \end{split}
\end{equation}
Here  also we may ignore $x^tx$

 \begin{equation} 
\begin{split}
g_k(x)  =\frac{ \mu_{k}^tx }{2 \sigma^2} + \frac{ \mu_{k}^t\mu_{k}}{2 \sigma^2} +  \ln(loss_{k}) + \ln(p(k))\\
 \end{split}
\end{equation}
Which is again a linear equation in x, So taking the decision boundary and writing in  norm of a line form \begin{equation} \nonumber
\begin{split}
w^t(x-x_0) = 0\end{split}
\end{equation} , we get 
 \begin{equation} 
\begin{split}
g_k(x)  =\frac{ \mu_{k}^tx }{2 \sigma^2} + \frac{ \mu_{k}^t\mu_{k}}{2 \sigma^2} +  \ln(loss_{k}) + \ln(p(k))\\
 \end{split}
\end{equation}

\begin{equation} \nonumber
\begin{split}
((u_1 - u_0)^t )x =(u_1 - u_0)^t )( \frac{\mu_1+\mu_0}{2} +(\mu_1-\mu_0)*\frac{ \sigma^2 \ln\frac{p*p(a)}{q*p(b)} }{(\mu_1 - \mu_0)^t (\mu_1 - \mu_0)}) \end{split}
\end{equation}
Geometrically,  this defines a hyperplane through the point $x_0$ that is orthogonal to the vector w. But since w= $\mu_1 - \mu_0$ then the hyperplane which seperates the regions is orthogonal to the line that links their means. If $p*p(a) = q*p(b)$, the second term on the right   vanishes, and thus the point $x_0$ is halfway between the means (equally divide the distance between the 2 means, with a decision region on either side), and the hy�perplane is the perpendicular bisector of the line between the means .

\section{Programming: Text Clustering}

In this problem, we will explore the use of EM algorithm for text
clustering. Text clustering is a technique for unsupervised document
organization, information retrieval. We want to find how to group a
set of different text documents based on their topics. First we will
analyze a model to represent the data.

\subsubsection*{Bag of Words}
The simplest model for text documents is to understand them as a
collection of words. To keep the model simple, we keep the
collection unordered, disregarding grammar and word order. What we
do is counting how often each word appears in each document and
store the word counts into a matrix, where each row of the matrix
represents one document. Each column of matrix represent a specific
word from the document dictionary. Suppose we represent the set of
$n_d$ documents using a matrix of word counts like this:

\begin{equation}
D_{1:n_d} =  \begin{pmatrix} 2&6&...&4\\ 2 & 4&...&0\\\vdots&
&\ddots \end{pmatrix} = T\nonumber
\end{equation}
This means that word $W_1$ occurs twice in document $D_1$ . Word
$W_{n_w}$ occurs 4 times in document $D_1$ and not at all in
document $D_2$.

\subsubsection*{Multinomial Distribution}

The simplest distribution representing a text document is
multinomial distribution(Bishop Chapter 2.2). The probability of a
document $D_i$ is:

\begin{equation}
p(D_i) = \prod_{j=1}^{n_w} \mu_j^{T_{ij}}\nonumber
\end{equation}
Here, $\mu_j$ denotes the probability of a particular word in the
text being equal to $w_j$, $T_{ij}$ is the count of the word in
document. So the probability of document $D_1$ would be $p(D_1)=
\mu_1^2\cdot\mu_2^6\cdot...\cdot\mu_{n_w}^4\cdot$

\subsubsection*{Mixture of Multinomial Distributions}

In order to do text clustering, we want to use a mixture of
multinomial distributions, so that each topic has a particular
multinomial distribution associated with it, and each document is a
mixture of different topics. We define $p(c) = \pi_c $  as the
mixture coefficient of a document containing topic $c$, and each topic
is modeled by a multinomial distribution $p(D_i|c)$ with parameters
$\mu_{jc}$, then we can write each document as a mixture over topics
as


\begin{equation}
p(D_i) = \sum_{c=1}^{n_c} p(D_i|c)p(c) = \sum_{c=1}^{n_c} \pi_c
\prod_{j=1}^{n_w} \mu_{jc}^{T_{ij}}\nonumber
\end{equation}

\subsubsection*{EM for Mixture of Multinomials}

In order to cluster a set of documents, we need to fit this mixture
model to data. In this problem, the EM algorithm can be used for
fitting mixture models. This will be a simple topic model for
documents. Each topic is a multinomial distribution over words (a
mixture component). EM algorithm for such a topic model, which
consists of iterating the following steps:

\begin{enumerate}
\item Expectation

Compute the expectation of document $D_i$ belonging to cluster $c$:

\begin{equation}
\gamma =  \frac{\pi_c \prod_{j=1}^{n_w}
\mu_{jc}^{T_{ij}}}{\sum_{c=1}^{n_d} \pi_c \prod_{j=1}^{n_w}
\mu_{jc}^{T_{ij}}}\nonumber
\end{equation}

\item Maximization

Update the mixture parameters, i.e. the probability of a word being
$W_j$ in cluster (topic) $c$, as well as prior probability of each
cluster.


\begin{equation}
\mu_{jc} =
\frac{\sum_{i=1}^{n_d}\gamma_{ic}T_{ij}}{\sum_{i=1}^{n_d}\sum_{l=1}^{m_w}
\gamma_{ic}T_{il}}\nonumber
\end{equation}

\begin{equation}
 \pi_c = \frac{1}{n_d}
\sum_{i=1}^{n_d}\gamma_{ic}\nonumber
\end{equation}

\end{enumerate}


\subsubsection*{Task [20
pts]}

Implement the algorithm and run on the toy dataset \texttt{data.mat}. You
can find detailed description about the data in the
\texttt{homework2.m} file. Observe the results and compare them with
the provided true clusters each document belongs to. Report the
evaluation (e.g. accuracy) of your implementation.

\emph{Hint:} We already did the word counting for you, so the data
file only contains a count matrix like the one shown above. For the toy dataset, set the
number of clusters $n_c = 4$. You will need to initialize the
parameters. Try several different random initial values for the
probability of a word being $W_j$ in topic $c$, $\mu_{jc}$. Make
sure you normalized it. Make sure that you should not use the true
cluster information during your learning phase.

\subsection*{Answer} Here, I am running the instance for 100 iterations(should be reduced larger datasets).Files used are homework2.m(running file) and mycluster.m(code for EM). I initialize the variables using repmat on random normal gaussian distribution across column/row , so that they are greater than 0 and sum to 1  in a row/column depending upon the requirement. I am getting from 80 to 90 percent accuracy for the above toy data set, which fluctuates depending upon the initializations for the prior probabilities.
Code is vectored, however commented code is based on loops for the understanding of the code.
\subsubsection*{Extra Credit: Realistic Topic Models [20pts]}
The above model assumes all the words in a document belongs to some topic at the same time. However, in real world datasets, it is more likely that
some words in the documents belong to one topic while other words belong to some other topics. For example, in a news report, some words may talk about ``Ebola'' and ``health'', while others may mention ``administration'' and ``congress''. In order to model this phenomenon, we should model each word as a mixture of possible topics.

Specifically, consider the log-likelihood of the joint distribution of document and words
\begin{equation}
\mathcal{L} = \sum_{d\in \mathcal{D}}\sum_{w \in \mathcal{W}} T_{dw} \log P(d, w),
\end{equation}
where $T_{dw}$ is the counts of word $w$ in the document $d$. This count matrix is provided as input.

The joint distribution of a specific document and a specific word is modeled as a mixture
\begin{equation}
P(d, w) = \sum_{z \in \mathcal{Z}} P(z) P(w|z) P(d|z),
\end{equation}
where $P(z)$ is the mixture proportion, $P(w|z)$ is the distribution over the vocabulary for the $z$-th topic, and $P(d|z)$ is the probability of the document for the $z$-th topic. And these are the parameters for the model.

The E-step calculates the posterior distribution of the latent variable conditioned on all other variables
\begin{equation}
P(z|d, w) = \frac{P(z) P(w|z) P(d|z)}{\sum_{z'}P(z') P(w|z') P(d|z')}.
\end{equation}

In the M-step, we maximizes the expected complete log-likelihood with respect to the parameters, and get the following update rules
\begin{align}
P(w|z) &= \frac{\sum_{d} T_{dw} P(z|d,w)}{\sum_{w'}\sum_{d} T_{dw'} P(z|d,w')} \\
P(d|z) &=  \frac{\sum_{w} T_{dw} P(z|d,w)}{\sum_{d'}\sum_{w} T_{d'w} P(z|d',w)} \\
P(z) &= \frac{\sum_{d}\sum_{w} T_{dw} P(z|d,w)}{\sum_{z'}\sum_{d'}\sum_{w'} T_{d'w'} P(z'|d',w')} .
\end{align}

\subsection*{Task}
 Implement EM for maximum likelihood estimation and cluster the
text data provided in the \texttt{nips.mat} file you downloaded. You can print out the top key words for the topics/clusters 
by using the \texttt{show\_topics.m} utility. It takes two parameters: 1) your learned conditional distribution matrix, i.e., $P(w|z)$ and
2) a cell array of words that corresponds to the vocabulary. You can find the cell array \texttt{wl} in the \texttt{nips.mat} file.
Try different values of $k$ and see which values produce sensible topics. In assessing your code, we will use another dataset and 
observe the produces topics. \\[.20cm]
\subsection*{Answer} Files for running this part are extrahomework2.m (running file) and topicmycluster.m (contains code for generating the topics). Here I have tried getting clusters/topics corresponding to different values of K. Like in the data set provided, too less clusters example 2 gave confusing topics as it kind of merged the topics. I was getting reasonable topics when I was using 4 to 6 clusters . For clusters greater than 10, it seemed that similar topics are getting distributed in different clusters.Code is vectored, however commented code is based on loops for the understanding of the code.Currently, I am running for 25 iterations.
%\bibliographystyle{plain}
%\bibliography{temp,externalPapers,groupPapers}

\end{document}
